%==============================================================================
% APRESENTAÇÃO BEAMER - VERSÃO FINAL COM TABELAS
%==============================================================================

\documentclass{beamer}

%==============================================================================
% 1. TEMA E ESTRUTURA VISUAL
%==============================================================================
\usetheme{metropolis}

%==============================================================================
% 2. PACOTES ESSENCIAIS (PREÂMBULO FINAL E LIMPO)
%==============================================================================
\usepackage{fontspec}
\usepackage{booktabs} % Para tabelas com aspeto profissional
\usepackage{multicol} % Para criar colunas de texto

%==============================================================================
% 3. CONFIGURAÇÃO DE FONTES
%==============================================================================
\usepackage[light]{sourcesanspro}

%==============================================================================
% 4. PALETA DE CORES E IDENTIDADE VISUAL
%==============================================================================
\definecolor{crpPrincipal}{HTML}{422C73}
\definecolor{crpSecundaria2}{HTML}{BF73AB}

\setbeamercolor{normal text}{fg=crpPrincipal, bg=white}
\setbeamercolor{progress bar}{fg=crpSecundaria2, bg=crpPrincipal!20}
\setbeamercolor{frametitle}{bg=crpPrincipal, fg=white}
\setbeamercolor{alerted text}{fg=crpSecundaria2}
\setbeamercolor{subtitle}{fg=crpPrincipal}

%==============================================================================
% 5. METADADOS DO DOCUMENTO
%==============================================================================
\title{Publicações impressas e digitais}
\subtitle{\textcolor{crpPrincipal}{Livros, cartilhas, Jornal Psi}}
\author{Conselho Regional de Psicologia de São Paulo}
\date{Outubro de 2025}

%==============================================================================
% INÍCIO DO DOCUMENTO
%==============================================================================
\begin{document}

\begin{frame}[plain]
    \titlepage
\end{frame}

\part{Análise e cenários}
\section{Histórico}

%------------------------------------------------------------------------------
% SLIDE 1: TABELA DE PRODUÇÃO ANUAL
%------------------------------------------------------------------------------
\begin{frame}[fragile]{Tabela 1: Produção Anual por Categoria}
    \framesubtitle{Número de volumes publicados por ano}
    \tiny % Reduz o tamanho da fonte para a tabela caber
    \begin{tabular}{l c c c c c}
        \toprule
        \textbf{Ano} & \textbf{Cadernos} & \textbf{Cartilhas} & \textbf{Livros} & \textbf{Diversos} & \textbf{Rev. Práticas Psi} \\
        \midrule
        2025 & & 3 & & 5 & 1 \\
        2022 & 2 & & 1 & 7 & \\
        2021 & & & & 1 & \\
        2020 & 1 & & & 4 & \\
        2019 & 16 & 2 & 9 & 123\textsuperscript{*} & \\
        2017 & & & 2 & & \\
        2016 & 6 & 1 & 8 & & \\
        2015 & 2 & 1 & & & \\
        2014 & & 3 & 1 & & \\
        2013 & & & 2 & & \\
        2012 & 1 & & & & \\
        2011 & 3 & & & & \\
        2010 & 3 & 2 & 2 & & \\
        2009 & 1 & & & & \\
        2008 & 1 & & & & \\
        2007 & 5 & & & & \\
        2006 & & & 1 & & \\
        2002 & & & & 1 & \\
        \bottomrule
    \end{tabular}
    
    \vspace{1em}
    \scriptsize % Aumenta ligeiramente a fonte para a nota
    \textsuperscript{*} Em 2019, a categoria 'Diversos' teve uma produção atípica de 123 volumes, incluindo uma vasta coleção de relatórios de inspeção.
\end{frame}

%------------------------------------------------------------------------------
% SLIDE 2: TABELA DE SÉRIES EDITORIAIS
%------------------------------------------------------------------------------
\begin{frame}{Tabela 2: Período das Séries Editoriais}
    \begin{tabular}{l l}
        \toprule
        \textbf{Série Editorial} & \textbf{Período} \\
        \midrule
        Cadernos Temáticos & 2007–2025 \\
        Comunicação Popular & 2010–2016 \\
        Pioneiras da Psicologia & 2022 \\
        Qualificação Profissional & 2019 \\
        Caderno de Debates & 2016 \\
        Em Debate & 2012 \\
        \bottomrule
    \end{tabular}
\end{frame}

%------------------------------------------------------------------------------
% SLIDE 3: TABELA DO HISTÓRICO DO JORNAL PSI
%------------------------------------------------------------------------------
\begin{frame}{Tabela 3: Histórico do Jornal Psi}
    \framesubtitle{Volumes publicados por ano (1981-2025)}
    \begin{multicols}{2}
    \tiny
    \begin{tabular}{l r}
        \toprule
        \textbf{Ano} & \textbf{Volumes} \\
        \midrule
        2025 & 2 \\
        2024 & 4 \\
        2023 & 2 \\
        2022 & 3 \\
        2021 & 1 \\
        2020 & 1 \\
        2019 & 4 \\
        2018 & 3 \\
        2017 & 1 \\
        2016 & 5 \\
        2015 & 5 \\
        2014 & 5 \\
        2013 & 5 \\
        2012 & 2 \\
        2011 & 3 \\
        2010 & 6 \\
        2009 & 6 \\
        2008 & 3 \\
        2007 & 6 \\
        2006 & 5 \\
        2005 & 4 \\
        2004 & 4 \\
        \bottomrule
    \end{tabular}
    
    \columnbreak
    
    \begin{tabular}{l r}
        \toprule
        \textbf{Ano} & \textbf{Volumes} \\
        \midrule
        2003 & 4 \\
        2002 & 5 \\
        2001 & 6 \\
        2000 & 5 \\
        1999 & 6 \\
        1998 & 6 \\
        1997 & 7 \\
        1996 & 7 \\
        1995 & 7 \\
        1994 & 7 \\
        1993 & 5 \\
        1992 & 6 \\
        1991 & 7 \\
        1990 & 6 \\
        1989 & 7 \\
        1988 & 8 \\
        1987 & 3 \\
        1986 & 8 \\
        1985 & 8 \\
        1984 & 13 \\
        1983 & 9 \\
        1982 & 6 \\
        1981 & 7 \\
        \bottomrule
    \end{tabular}
    \end{multicols}
\end{frame}


%------------------------------------------------------------------------------
% O RESTO DO SEU DOCUMENTO
%------------------------------------------------------------------------------
\section{Contexto atual}
\begin{frame}{Linhas editoriais ativas}
    \normalsize
    \textbf{Categorias no \textit{site}}
    
    \scriptsize
    \begin{itemize}
        \item Cadernos Temáticos
        \item Cartilhas
        \item Crepop
        \item Jornal Psi
        \item Livros
        \item Memória da Psicologia
        \item Revista Práticas Psi
        \item Diversos
    \end{itemize}
    
    \vspace{0.5em}
    
    \normalsize
    \textbf{Divisão conceitual}
    
    \scriptsize
    \begin{itemize}
        \item \textbf{Institucional e memória:} inclui premiações, relatórios de gestão, anais de eventos etc.
        \item \textbf{Informação e divulgação:} Jornal Psi, Cadernos Temáticos, Revista Práticas Psi
        \item \textbf{Orientação e normativas:} cartilhas, carta de serviços, manuais etc.
    \end{itemize}
\end{frame}

\section{Exploração de Cenários}
\begin{frame}{Exploração de cenários}
\end{frame}

\part{Propostas e Encaminhamentos}
\section{Linhas Editoriais}
\begin{frame}{Sugestão de Linhas Editoriais}
\end{frame}

\section{Sugestões para o Jornal Psi}
\begin{frame}{Sugestões para o Jornal Psi}
\end{frame}

\section{Encaminhamentos}
\begin{frame}{Encaminhamentos Necessários}
\end{frame}

\end{document}