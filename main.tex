\documentclass{beamer}

% --- 1. TEMA ESTRUTURAL ---
\usetheme{metropolis}

% --- 2. PACOTES ---
\usepackage{fontspec}
\usepackage{tikz}
\usepackage{pgfplots}
\usepackage{pgfplotstable}
\pgfplotsset{compat=1.18}
\usepgfplotslibrary{groupplots} % Necessário para xbar interval

% --- 3. FONTES INSTITUCIONAIS ---
\setsansfont{NEWJUNE-REGULAR.OTF}[
    Path           = ./fonts/,
    UprightFont    = NEWJUNE-REGULAR.OTF,
    ItalicFont     = NEWJUNE-REGULAR-ITALIC.OTF,
    BoldFont       = NEWJUNE-BOLD.OTF,
    BoldItalicFont = NEWJUNE-BOLD-ITALIC.OTF
]

% --- 4. CORES E ESTILOS PERSONALIZADOS ---
\definecolor{crpPrincipal}{HTML}{422C73}
\definecolor{crpSecundaria2}{HTML}{BF73AB}
\setbeamercolor{normal text}{fg=crpPrincipal, bg=white}
\setbeamercolor{progress bar}{fg=crpSecundaria2, bg=crpPrincipal!20}
\setbeamercolor{frametitle}{bg=crpPrincipal}
\addtobeamertemplate{frametitle}{\color{white}}{\color{white}}
\setbeamercolor{alerted text}{fg=crpSecundaria2}
\setbeamercolor{subtitle}{fg=crpPrincipal}

% --- 5. LÓGICA DE PROCESSAMENTO DE DADOS COM LUA ---
% Esta é a abordagem robusta para mapear texto para números.
\directlua{
function map_category_to_y(cat)
    if cat == "Cadernos Temáticos" then return 1
    elseif cat == "Cartilhas" then return 2
    elseif cat == "Jornal Psi" then return 3
    elseif cat == "Livros" then return 4
    elseif cat == "Diversos" then return 5
    elseif cat == "Revista Práticas Psi" then return 6
    else return 0 -- Categoria desconhecida
    end
end
}

% --- 6. DEFINIÇÃO E PROCESSAMENTO DAS TABELAS DE DADOS ---
% Leitura dos dados brutos
\pgfplotstableread[col sep=comma, header=true]{
categoria, ano, volumes
Cadernos Temáticos, 2022, 2
Cadernos Temáticos, 2020, 1
Cadernos Temáticos, 2019, 16
Cadernos Temáticos, 2016, 6
Cadernos Temáticos, 2015, 2
Cadernos Temáticos, 2012, 1
Cadernos Temáticos, 2011, 3
Cadernos Temáticos, 2010, 3
Cadernos Temáticos, 2009, 1
Cadernos Temáticos, 2008, 1
Cadernos Temáticos, 2007, 5
Cartilhas, 2025, 3
Cartilhas, 2019, 2
Cartilhas, 2016, 1
Cartilhas, 2015, 1
Cartilhas, 2014, 3
Cartilhas, 2010, 2
Diversos, 2025, 5
Diversos, 2022, 7
Diversos, 2021, 1
Diversos, 2020, 4
Diversos, 2019, 123
Diversos, 2002, 1
Jornal Psi, 2025, 2
Jornal Psi, 2024, 4
Jornal Psi, 2023, 2
Jornal Psi, 2022, 3
Jornal Psi, 2021, 1
Jornal Psi, 2020, 1
Jornal Psi, 2019, 4
Jornal Psi, 2018, 3
Jornal Psi, 2017, 1
Jornal Psi, 2016, 5
Jornal Psi, 2015, 5
Jornal Psi, 2014, 5
Jornal Psi, 2013, 5
Jornal Psi, 2012, 2
Jornal Psi, 2011, 3
Jornal Psi, 2010, 6
Jornal Psi, 2009, 6
Jornal Psi, 2008, 3
Jornal Psi, 2007, 6
Jornal Psi, 2006, 5
Jornal Psi, 2005, 4
Jornal Psi, 2004, 4
Jornal Psi, 2003, 4
Jornal Psi, 2002, 5
Jornal Psi, 2001, 6
Jornal Psi, 2000, 5
Jornal Psi, 1999, 6
Jornal Psi, 1998, 6
Jornal Psi, 1997, 7
Jornal Psi, 1996, 7
Jornal Psi, 1995, 7
Jornal Psi, 1994, 7
Jornal Psi, 1993, 5
Jornal Psi, 1992, 6
Jornal Psi, 1991, 7
Jornal Psi, 1990, 6
Jornal Psi, 1989, 7
Jornal Psi, 1988, 8
Jornal Psi, 1987, 3
Jornal Psi, 1986, 8
Jornal Psi, 1985, 8
Jornal Psi, 1984, 13
Jornal Psi, 1983, 9
Jornal Psi, 1982, 6
Jornal Psi, 1981, 7
Livros, 2022, 1
Livros, 2019, 9
Livros, 2017, 2
Livros, 2016, 8
Livros, 2014, 1
Livros, 2013, 2
Livros, 2010, 2
Livros, 2006, 1
Revista Práticas Psi, 2025, 1
}\publicacoes

\pgfplotstableread[col sep=comma, header=true]{
serie, y_cat, inicio, fim
Em Debate, 5, 2012, 2012
Caderno de Debates, 5, 2016, 2016
Qualificação Profissional, 5, 2019, 2019
Pioneiras da Psicologia, 5, 2022, 2022
Comunicação Popular, 5, 2010, 2016
Cadernos Temáticos, 1, 2007, 2025
}\series

% Usa a função Lua para criar a coluna de coordenadas Y de forma segura.
\pgfplotstablecreatecol[create col/lua={
    categoria = table.getraw("categoria")
    return map_category_to_y(categoria)
}]{ycoord}{\publicacoes}

% --- INFORMAÇÕES DO DOCUMENTO ---
\title{Publicações impressas e digitais}
\subtitle{\textcolor{crpPrincipal}{Livros, cartilhas, Jornal Psi}}
\author{Conselho Regional de Psicologia de São Paulo}
\date{Outubro de 2025}

% --- INÍCIO DO DOCUMENTO ---
\begin{document}

\begin{frame}[plain]
    \titlepage
\end{frame}

% === PRIMEIRA PARTE ===
\part{Análise e cenários}

\section{Histórico}
\begin{frame}{Histórico das publicações}
    \begin{tikzpicture}
        \begin{axis}[
            width=\textwidth, 
            height=0.8\textheight,
            title={Distribuição anual e períodos das linhas editoriais},
            font=\sffamily\small,
            title style={font=\sffamily\bfseries},
            label style={font=\sffamily\scriptsize},
            tick label style={font=\sffamily\tiny},
            xlabel={Ano de Publicação},
            xmin=1980, xmax=2026,
            xtick distance=5,
            minor xtick distance=1,
            ylabel={},
            y dir=reverse,
            ytick={1,2,3,4,5,6},
            yticklabels={Cadernos Temáticos, Cartilhas, Jornal Psi, Livros, Diversos, Revista Práticas Psi},
            yticklabel style={align=right, font=\sffamily\bfseries\tiny, xshift=-5pt},
            y tick label style={text width=2.5cm, align=right},
            axis line style={draw=none},
            tick style={draw=none},
            grid=major,
            grid style={thin, dashed, color=crpPrincipal!20},
            enlarge y limits={abs=0.8},
        ]

        % --- Camada 1: Duração das Séries (Contexto) ---
        \addplot+[
            xbar interval,
            fill=crpPrincipal!10,
            draw=none,
            bar width=10pt,
            bar shift=0pt
        ] table [x=inicio, x end=fim, y=y_cat] {\series};
        
        \foreach \i in {0,...,5}{
            \pgfplotstablegetrow{\i}{\series}
            \node[anchor=west, font=\sffamily\itshape\tiny, color=black!60] 
                at (axis cs:{\pgfplotstablevalueof{inicio}{\series}},{\pgfplotstablevalueof{y_cat}{\series}-0.3}) 
                {\pgfplotstablevalueof{serie}{\series}};
        }
        
        % --- Camada 2: Volume de Publicação (Dados) ---
        \addplot[
            scatter, only marks,
            scatter src=explicit symbolic,
            mark=*, color=crpPrincipal,
            scatter/use mapped size=true,
            scatter/mapped size={2}{20},
            nodes near coords*={\pgfmathprintnumber[int detect]{\plotpointmeta}},
            nodes near coords style={font=\sffamily\tiny, anchor=west, color=crpPrincipal!80!black},
        ] table [x=ano, y=ycoord, meta=volumes] {\publicacoes};

        \end{axis}
    \end{tikzpicture}
\end{frame}

\section{Contexto atual}
\begin{frame}{Linhas editoriais ativas}
    \normalsize
    \textbf{Categorias no \textit{site}}
    \scriptsize
    \begin{itemize}
        \item Cadernos Temáticos;
        \item Cartilhas;
        \item Crepop;
        \item Jornal Psi;
        \item Livros;
        \item Memória da Psicologia;
        \item Revista Práticas Psi;
        \item Diversos.
    \end{itemize}
    \normalsize
    \textbf{Divisão conceitual}
    \scriptsize
    \begin{itemize}
        \item \textbf{Institucional e memória:} inclui premiações, relatórios de gestão, anais de eventos etc.;
        \item \textbf{Informação e divulgação:} Jornal Psi, Cadernos Temáticos, Revista Práticas Psi;
        \item \textbf{Orientação e normativas:} cartilhas, carta de serviços, manuais etc.
    \end{itemize}
\end{frame}

\section{Exploração de Cenários}
\begin{frame}{Exploração de cenários}
    % Conteúdo do slide aqui
\end{frame}

% === SEGUNDA PARTE ===
\part{Propostas e Encaminhamentos}

\section{Linhas Editoriais}
\begin{frame}{Sugestão de Linhas Editoriais}
    % Conteúdo do slide aqui
\end{frame}

\section{Sugestões para o Jornal Psi}
\begin{frame}{Sugestões para o Jornal Psi}
    % Conteúdo do slide aqui
\end{frame}

\section{Encaminhamentos}
\begin{frame}{Encaminhamentos Necessários}
    % Conteúdo do slide aqui
\end{frame}

\end{document}